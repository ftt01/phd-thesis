%!TEX root = ../thesis.tex
%*******************************************************************************
%****************************** Third Chapter **********************************
%*******************************************************************************
\chapter{Reference local data}

% **************************** Define Graphics Path **************************
\ifpdf
    \graphicspath{{Chapter3/Figs/Raster/}{Chapter3/Figs/PDF/}{Chapter3/Figs/}}
\else
    \graphicspath{{Chapter3/Figs/Vector/}{Chapter3/Figs/}}
\fi

%********************************** %Section 3.1 **************************************
\section{High resolution gridded dataset of hourly precipitation and temperature (1991-2021) for the Trentino-Alto Adige\label{section3.1}}

% models in general

% \begin{itemize}
%     \item input uncertainties
%     \item parameters uncertainties
%     \item models structure uncertainties
% \end{itemize}

% next subsections: hydrological models and econometric models

%********************************** %Section 3.2 **************************************
% \section{Hydrological models
% \label{section3.2}}

% \subsection{Hydrological models and machine learning}
% \begin{itemize}
%     \item hydrological models in general: goals and state-of-art
%     \item complexity in prediction of the streamflow at high temporal resolution [non-stationarity, dynamism and non-linearity]
%     \item machine learning as hydrological models alternative: pros and cons
% \end{itemize}

% \subsection{A comparison: the Passirio case study}

% \subsection{An application: the Alto Adige case study}
% % 
% \begin{itemize}
%     \item PROBLEM: could data-driven methods be applied on Alto Adige?
%     \item METHODS: SVR on 20 basins
% \end{itemize}
% % 

% % 
% \subsection{A multi-model approach
% \label{section.sub.3.2.4:multimodel}}
% % 
% \begin{itemize}
%     \item structural uncertainties of the different available packages in Python
%     \item available methods in literature
%         \begin{itemize}
%             \item linear models: Auto Regressive (AR), Moving Average (MA), ARMA, ARIMA, ARIMAX, Linear Regression (LR) and Multiple Linear Regression (MLR) >> not able to catch complexity in streamflow forecasting
%             \item 
%             \item 
%         \end{itemize}
%     \item structural uncertainties due to the limits of the different methods
%     \item Application leveraging Multivariate LSTM vs SVR
% \end{itemize}

% \subsubsection{Introduction to main data-driven methods}


% \subsubsection{Multivariate LSTM}




% %********************************** %Section 3.3 **************************************
% \section{Econometric models\label{section3.3}}
% \subsection{Introduction to econometric models}
% \begin{itemize}
%     \item econometric models in general: goals and state-of-art
%     \item the forecasting in North Italy > description of the 58 methods implemented by Ravazzolo
% \end{itemize}














% \subsection{First subsection in the first section}

% \subsubsection{First subsub section in the third subsection}
% \dots and some more in the first subsub section otherwise it all looks the same
% doesn't it? well we can add some text to it and some more and some more and
% some more and some more and some more and some more and some more \dots

% \subsubsection{Second subsub section in the third subsection}
% \dots and some more in the first subsub section otherwise it all looks the same
% doesn't it? well we can add some text to it \dots

% \section{Second section of the third chapter}
% and here I write more \dots

% \section{The layout of formal tables}
% This section has been modified from ``Publication quality tables in \LaTeX*''
%  by Simon Fear.

% The layout of a table has been established over centuries of experience and 
% should only be altered in extraordinary circumstances. 

% When formatting a table, remember two simple guidelines at all times:

% \begin{enumerate}
%   \item Never, ever use vertical rules (lines).
%   \item Never use double rules.
% \end{enumerate}

% These guidelines may seem extreme but I have
% never found a good argument in favour of breaking them. For
% example, if you feel that the information in the left half of
% a table is so different from that on the right that it needs
% to be separated by a vertical line, then you should use two
% tables instead. Not everyone follows the second guideline:

% There are three further guidelines worth mentioning here as they
% are generally not known outside the circle of professional
% typesetters and subeditors:

% \begin{enumerate}\setcounter{enumi}{2}
%   \item Put the units in the column heading (not in the body of
%           the table).
%   \item Always precede a decimal point by a digit; thus 0.1
%       {\em not} just .1.
%   \item Do not use `ditto' signs or any other such convention to
%       repeat a previous value. In many circumstances a blank
%       will serve just as well. If it won't, then repeat the value.
% \end{enumerate}

% A frequently seen mistake is to use `\textbackslash begin\{center\}' \dots `\textbackslash end\{center\}' inside a figure or table environment. This center environment can cause additional vertical space. If you want to avoid that just use `\textbackslash centering'


% \begin{table}
% \caption{A badly formatted table}
% \centering
% \label{table:bad_table}
% \begin{tabular}{|l|c|c|c|c|}
% \hline 
% & \multicolumn{2}{c}{Species I} & \multicolumn{2}{c|}{Species II} \\ 
% \hline
% Dental measurement  & mean & SD  & mean & SD  \\ \hline 
% \hline
% I1MD & 6.23 & 0.91 & 5.2  & 0.7  \\
% \hline 
% I1LL & 7.48 & 0.56 & 8.7  & 0.71 \\
% \hline 
% I2MD & 3.99 & 0.63 & 4.22 & 0.54 \\
% \hline 
% I2LL & 6.81 & 0.02 & 6.66 & 0.01 \\
% \hline 
% CMD & 13.47 & 0.09 & 10.55 & 0.05 \\
% \hline 
% CBL & 11.88 & 0.05 & 13.11 & 0.04\\ 
% \hline 
% \end{tabular}
% \end{table}

% \begin{table}
% \caption{A nice looking table}
% \centering
% \label{table:nice_table}
% \begin{tabular}{l c c c c}
% \hline 
% \multirow{2}{*}{Dental measurement} & \multicolumn{2}{c}{Species I} & \multicolumn{2}{c}{Species II} \\ 
% \cline{2-5}
%   & mean & SD  & mean & SD  \\ 
% \hline
% I1MD & 6.23 & 0.91 & 5.2  & 0.7  \\

% I1LL & 7.48 & 0.56 & 8.7  & 0.71 \\

% I2MD & 3.99 & 0.63 & 4.22 & 0.54 \\

% I2LL & 6.81 & 0.02 & 6.66 & 0.01 \\

% CMD & 13.47 & 0.09 & 10.55 & 0.05 \\

% CBL & 11.88 & 0.05 & 13.11 & 0.04\\ 
% \hline 
% \end{tabular}
% \end{table}


% \begin{table}
% \caption{Even better looking table using booktabs}
% \centering
% \label{table:good_table}
% \begin{tabular}{l c c c c}
% \toprule
% \multirow{2}{*}{Dental measurement} & \multicolumn{2}{c}{Species I} & \multicolumn{2}{c}{Species II} \\ 
% \cmidrule{2-5}
%   & mean & SD  & mean & SD  \\ 
% \midrule
% I1MD & 6.23 & 0.91 & 5.2  & 0.7  \\

% I1LL & 7.48 & 0.56 & 8.7  & 0.71 \\

% I2MD & 3.99 & 0.63 & 4.22 & 0.54 \\

% I2LL & 6.81 & 0.02 & 6.66 & 0.01 \\

% CMD & 13.47 & 0.09 & 10.55 & 0.05 \\

% CBL & 11.88 & 0.05 & 13.11 & 0.04\\ 
% \bottomrule
% \end{tabular}
% \end{table}
